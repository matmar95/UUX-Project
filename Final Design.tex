\documentclass[12pt,a4paper]{report}
\usepackage[italian]{babel}
\usepackage[T1]{fontenc}
\usepackage[sfdefault]{noto}
\usepackage{graphicx}
\usepackage{multirow}
\usepackage{enumitem}
\usepackage{hyperref}
\hypersetup{pdfborder = 0 0 0 }
\usepackage{wrapfig}
\usepackage{color}
\linespread{1.3}
\textwidth=450pt\oddsidemargin=0pt
\begin{document}
\begin{titlepage}
\vspace{15mm}
\begin{center}
  \includegraphics{"Images Latex/Project Management Report/UniBo-Universita-di-Bologna"}
\end{center}
\begin{center}
{\normalsize{\bf Corso di Laurea Magistrale in Informatica}}\\
\vspace{5mm}
{\Large{\bf Progetto di Usability and User Experience}}\\
\vspace{5mm}
{\normalsize{\bf Anno Accademico 2017/2018}}\\
\vspace{20mm}
{\normalsize{\bf Final Design}}\\
\vspace{5mm}
{\Huge{\bf Little Women}}\\
\vspace{5mm}
{\Large{\bf More than a Girl}}\\
\vspace{25mm}
\end{center}
\begin{flushright}
{\large{Matteo Sanfelici\\0000856403\\matteo.sanfelici@studio.unibo.it\\}}
\vspace{5mm}
{\large{Matteo Marchesini\\0000856336\\matteo.marchesini12@studio.unibo.it}}
\end{flushright}
\end{titlepage}
\tableofcontents
\chapter{Introduzione}
Al giorno d'oggi per un'azienda impegnata nel commercio retail, un sito di e-commerce è un'opportunità di crescita sia per l'azienda che per un cliente.\\
Ormai le aziende impegante nella vendita di prodotti su internet iniziano ad essere numerose nonostante ci siano costi fissi da sostenere non inferiori a quelli del commercio fisico in negozio.\\
L'acquisto online di un prodotto sembra un'operazione semplice o banale al giorno d'oggi. Questa semplicità non è dovuta all'operazione in se, ma piuttosto al fatto che ogni azione che si può intraprendere su di un sito di shopping online è stata studiata e progettata con cura al fine di essere il più intuitiva possibile. Lo studio che viene effettuato per sviluppare un sito altamente usabile implica l'utilizzo di teorie di Usabilità che tengano conto dei diversi ruoli e differenti utenti a cui il sito in questione è rivolto.\\
Abbiamo quindi progettato un sito che permette ad un'azienda di abbigliamento (e prodotti di moda) di lanciarsi sul web e offrire un servizio i vendita sicuro, affidabile e soprattuto semplice e intuitivo da usare.\\
Durante tutta la progettazione, come di comune accordo col committente, è sempre stato centrale i prodotti da vendere e il target di utenti a cui ci si rivolgeva: vendita di abbigliamento, scarpe e accessori a ragazze dai 13 ai 16 anni (\textit{plot}).\\
Tentndo conto del target, sono state effettuate scelte progettuali che favoriscono e semplificano alcune operazioni macchinose per mancanze di competenze nell'uso del medium (internet) o del dominio (moda).\\
Il cliente ha anche rimarcato il suo particolare interesse per le tematiche della sicurezza dei minori online e dell'impiego di politiche parent-friendly. In pratica il cliente ritiene che la fase di scelta, ma soprattuto di acquisto debba essere fatta di comune accordo e sotto la supervisione di un adulto. Perciò abbiamo dovuto tener conto anche della presenza di un utente secondario (\textit{subplot}), cioè i genitori delle ragazze o comunque adulti responsabili per loro.\\
Plot e Subplot si intrecciano tra loro con obiettivi e prospettivi differenti. La progettazione del sito ha portato alla creazione di sezioni riservate ai soli genitori, dividendo l'utenze in due tipologie di account, una per le ragazze (con meno operazioni possibili) e uno per i genitori (con il totale controllo della fase di pagamento o reso ad esempio).\\

\chapter{Ricerca Etnografica}
Il sito LittleWomen è stato progettato con lo scopo di fornire una piattaforma di e-commerce per la vendita di prodotti di abbigliamento scarpe e accessori. Il target d'utenza, come concordato con il cliente, risulta abbastanza ristretto, essendo composto da ragazze tra i 13 e i 16 anni (\textit{plot}); si approcciano all'acquisto di prodotti online e tavolta al mondo della moda per la prima volta, tuttavia sono ormai capaci di utilizzare il web in modo autonomo.\\
Il sito adotta politiche \textit{parent-friendly}, come richiesto dal cliente. Ciò significa che la progettazione ha dovuto tener conto dell'utente genitore, fornendogli i mezzi per fari si che le ragazze siano protette dai pericoli che possono accorrere durante la navigazione e l'acquisto di articoli su siti di e-commerce. \\
Per soddisfare tale necessità è stata ideata un'apposita sezione per i genitori nella quale hanno il privilegio di compiere determinati task senza i quali l'acquisto non sarebbe portato a termine. Infatti è necessario che l'account della ragazza sia collegato ad un account adulto, ovvero il genitore o chi ne fa le veci. Entrambe le classi di utenti potranno fare acquisti, ma la fase validazione degli acquisti nonchè di pagamento sarà riservata all'utente genitore, che dovrà confermare l'acquisto inserendo di persona i dati di pagamento. Nella pratica, la ragazza quando desidera acquistare un articolo lo aggiungerà al carrello, e nel carrello del relativo account adulto verranno visualizzati i medesimi prodotti. Spetterà poi al genitore accettare o rigettare gli acquisti effettuati dalla ragazza. Da tutto ciò ne deriva un controllo totale dell'utilizzo di una piattaforma di e-commerce da parte di minori.
\chapter{Blueprint}
\chapter{Wireframe}La proposta di design è stata realizzata tramite il software Balsamiq Mockups 3, che consenti di fare uno sketch di una qualsiasi interfaccia. Dopo la definizione della struttura del sito tramite il \textit{blueprint}, in questo capitolo ci concentriamo sulla descrizione più specifica delle singole pagina e delle feature principali.
\section{Versione Desktop}
\subsection{Navbar e Footer} Tutte le pagine del sito hanno in comune sia la Navbar in cime, che il Footer in basso.\\
La Navbar è centrale nella navigazione del sito in quanto raggruppa i principali comandi e shortcut per raggiungere le principali sezioni del sito. Sarà sempre fissata in cima alla pagina anche se si scrolla di molto verso il basso.\\
\includegraphics[width=\textwidth]{"Images Latex/Immagini Wireframe/Desktop/1 - Navbar"}\\
È suddivisa in 3 parti principali:
\begin{itemize}
  \item Sezione relativa a login e registrazione e modifica della lingua del sito
  \item Sezione centrale con logo che riporta sempre alla Home, la barra di ricerca e le shortcut per l'accesso a profilo, preferiti e carrello
  \item Sezione menù con cui accedere alle macrocategorie del sito quali: Novità, Abbigliamento, Scarpe, Accessori e Offerte.
\end{itemize}
\includegraphics[width=\textwidth]{"Images Latex/Immagini Wireframe/Desktop/4 - Navbar Menu"}
\vspace{5mm}\\
Il Footer è l'altra parte di sito condivisa a tutte le pagine. Qua possiamo trovare alcune cose di importanza relativa per la navigazione del sito, come le informazioni relative a contatti e negozi fisici e alcuni link a pagine informative o di assitenza su spedizioni, resi e pagamenti.\\
\includegraphics[width=\textwidth]{"Images Latex/Immagini Wireframe/Desktop/1 - Footer"}
\newpage
\subsection{Homepage}
\begin{center}
\includegraphics[width=0.80\textwidth]{"Images Latex/Immagini Wireframe/Desktop/1 - Homepage"}
\end{center}
Esclusa la navbar che è presente in ogni schermata, la homepage è suddivisa in 2 sezioni.
\begin{itemize}
  \item Sezione Novità, dove si possono visualizzare le ultime novità in evidenza.
  \item Sezione Offerte, dove viene mostrato un estratto dei principali prodotti in offerta
\end{itemize}
In entrambe le sezioni ci son due bottoni:  \textit{Novità - le tendenze del momento} e \textit{Offerte}. Cliccando su di essi si passerà alla relativa macrocategoria.
\subsection{Aiuto e Assisteza}
\subsubsection*{Contatta l'assistenza}
\begin{center}
\includegraphics[width=\textwidth]{"Images Latex/Immagini Wireframe/Desktop/20 - Assistenza"}
\end{center}
In questa schermata raggiungibile dal footer cliccando su Assistenza, è presente un form da compilare per ricevere assistenza.\vspace{4mm}
\subsubsection*{Aiuto: Metodi di spedizione, Resi e rimborsi}
Nel footer sono sempre presenti due shortcut a due pagine utili a ricevere informazioni relative ai metodi di spedizione e alle politiche di reso/rimborso.
\newpage
\subsection{Menù di navigazione laterale}
\begin{wrapfigure}{l}{0.3\textwidth}
\vspace{-3mm}
  \includegraphics[height=0.5\textheight]{"Images Latex/Immagini Wireframe/Desktop/7 - Dettagli Prodotto"}
\end{wrapfigure}
Con il menù di navigazione laterale ci si può facilmente spostare tra le varie categorie cliccando sulla label relativa.\\
Scendendo in sottocategorie sarà sempre possibile visualizzare le categorie al macrocategorie al di sopra di dove ci si trova e tornare velocemente indietro.\\
\subsection{Dettagli Articolo}

\section{Versione Mobile}
\chapter{Conclusioni}
\chapter{Licenza}
\end{document}
