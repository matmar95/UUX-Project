\documentclass[12pt,a4paper]{report}
\usepackage[italian]{babel}
\usepackage[T1]{fontenc}
\usepackage[sfdefault]{noto}
\usepackage{graphicx}
\usepackage{multirow}
\usepackage{enumitem}
\usepackage{hyperref}
\hypersetup{pdfborder = 0 0 0 }
\usepackage{wrapfig}
\usepackage{color}
\linespread{1.3}
\textwidth=450pt\oddsidemargin=0pt
\begin{document}
\begin{titlepage}
\vspace{15mm}
\begin{center}
  \includegraphics{"Images Latex/Project Management Report/UniBo-Universita-di-Bologna"}
\end{center}
\begin{center}
{\normalsize{\bf Corso di Laurea Magistrale in Informatica}}\\
\vspace{5mm}
{\Large{\bf Progetto di Usability and User Experience}}\\
\vspace{5mm}
{\normalsize{\bf Anno Accademico 2017/2018}}\\
\vspace{20mm}
{\normalsize{\bf Project Management Report}}\\
\vspace{5mm}
{\Huge{\bf Little Women}}\\
\vspace{5mm}
{\Large{\bf More than a Girl}}\\
\vspace{25mm}
\end{center}
\begin{flushright}
{\large{Matteo Sanfelici\\0000856403\\matteo.sanfelici@studio.unibo.it\\}}
\vspace{5mm}
{\large{Matteo Marchesini\\0000856336\\matteo.marchesini12@studio.unibo.it}}
\end{flushright}
\end{titlepage}
\tableofcontents
\chapter{Introduzione}
LittleWomen è una proposta di un sito di e-commerce rivolto principalmente ad un'utenza di ragazze comprese tra i 13 e i 16 anni.\\Questo sito è stato progettato durante il corso di Usability and User Experience nell'ambito della traccia "More than a Girl".
\vspace{10mm}\\ Al giorno d'oggi per un'azienda che si occupa della vendita di prodotti al dettaglio è ormai indispensabile avere di un forte presenza on-line attraverso un sito dedicato di e-commerce.\\
È quindi importante offrire un servizio semplice e intuitivo ai propri clienti che desiderano acquistare prodotti in maniera sicura direttamente da casa.\\
Effettuare un acquisto online è ormai un'operazione facile da eseguire, questo però solo grazie ad una progettazione efficace da parte di designer ed esperti progettisti che danno grande rilievo ad ottenere una buona usabilità e offire un'ottima esperienza all'utente finale.
\vspace{10mm}\\Come stabilito con il committente, la progettazione ha dovuto tener conto dell'utente a cui sono rivolti i servizi offerti dal sito. Dato che l'azienda è importntata alla vendita di capi d'abbigliamento, calzature e accessori per ragazze comprese nella fascia d'età 13-16 anni, quest'ultime sono il target principale di utenza del sito (\textit{plot}).\\
In fase di accordo, il cliente ha espresso apprensione verso le tematiche della sicurezza dei minori online, adottando politiche \textit{parent-friendly}. In sostanza il cliente è dell'idea che la progettazione del sito debba soffermarsi anche sul ruolo del genitore che diventa parte integrante del processo di scelta e acquisto di un prodotto.
\vspace{10mm}\\Viene quindi individuato nel genitore o in un generico adulto un attore secondario (\textit{subplot}), il quale assume un ruolo chiave nell'interazione col sito.
Il genitore (o chi per lui) si occuperà di completare alcuni \textit{task} che richiedono la sua supervisione (e.g. il pagamento con carta di credito di un prodotto).\\
\vspace{10mm}\\Seguendo queste linee guida imposte dal committente, è stata individuata la possibilità di poter creare due tipi di account separati, uno per il minore e uno per l'adulto. L'account del minore dovrà essere associato a quello del proprio genitore, il quale dovrà completare l'acquisto dei prodotti selezionati dalla proprio figlia. Entrambi potranno comunque consultare il sito e aggiungere prodotti al carrello, ma la fase di pagamento sarà riservata all'account del genitore.
\vspace{10mm}\\Il progetto è strutturato in 5 fasi:
\begin{itemize}
  \item \textbf{Ricerca etnografica}: viene identificato il target di utenti a cui è rivolto il sistema e viene studiato attraverso ricerche di mercato e sondaggi, cercando di capire e individuare i bisogni degli utenti e i task che devono essere messi a disposizione.
  \item \textbf{Studio di fattibilità} viene studiato l'ipotetico contesto d'uso del sito, effettuando una profilazione degli utenti (personas) e dei loro bisogni e analizzando gli scenari di utilizzo del sistema.
  \item \textbf{Valutazione delle risorse esistenti} si valutano le risorse esistenti attraverso linee guida di usabilità e successivamente si conducono test insieme agli utenti.
  \item \textbf{Proposta di design} si elebora un possibile design di un sito attraverso blueprint e wireframes introducendo funzionalità volte a raggiungere l'obiettivo del progetto.
  \item \textbf{Valutazione del design} si valuta il prodotto allo stesso modo delle risorse esistenti, al fine di trovare e risolvere eventuali problemi di design.
\end{itemize}
\chapter{Ricerca etnografica}
La progettazione di un sito web si pone come punto di partenza la definizione dei bisogni che esso dovrà soddisfare. Ciò è reso possibile grazie all'analisi degli utenti e dei loro goal (obiettivi). La vendita di abbigliamento per ragazze è un settore molto complesso nonchè ricco di aziende già inserite da tempo, per cui sarà necessario soffermarsi sulle motivazioni che spingono una ragazza a compiere acquisti sul proprio sito web piuttosto che su un altro. L'individuazione delle motivazioni che comportano tale scelta può essere effettuata tramite una tecnica chiamata segmentazione, la quale permette di individuare l'esatto segmento di mercato che risponde ai parametri stabiliti in fase di commissione. Nella sezione 2.1 verranno approfonditi i segmenti individuati nonchè gli attori del sistema, quali ragazze e relativi genitori o adulti. Successivamente sono stati ricercati tramite sondaggi soggetti reali conformi ai criteri stabiliti in fase di segmentazione.
\section{Segmentazione}
Per la segmentazione degli utenti sono stati seguiti diversi criteri quali:
 età, condizione economica, competenze digitali, motivazioni e finalità.
\newpage
Sono state identificate le seguenti categorie:
 \begin{enumerate}
   \item \textbf{Ragazze tra i 13 e i 16 anni}
   \begin{enumerate}[label=\alph*.]
     \item \textbf{Tra 13 e 14 anni}\\
     Utenti ancora non maturi nell'utilizzo di siti di e-commerce, ma comunque capaci di utilizzare il web in modo autonomo dato che fanno parte della categoria dei nativi digitali.\\
     Data l'età, ancora non hanno ben chiaro ciò che cercano e lo stile che vogliono adottare, quindi necessitano di un sito web che le consigli sul modo di vestire e che offra molti spunti e alternative di stile da adottare.
     \item \textbf{Tra 15 e 16 anni}\\
     Utenti già consapevoli nell'utilizzo di siti di shopping online, dato che ci navigano da qualche anno. Sono anch'esse native digitali e quindi non hanno problemi nel muoversi all'interno di un sito e di internet in generale.\\
     Possono essere considerate già più mature dal punto di vista di uno stile personale e quindi potrebbero voler arrivare direttamente a un prodotto specifico, anche se non disdegnano la possibilità di vagliare alternative.
   \end{enumerate}
   \item \textbf{Genitori e Adulti}\\
   Genitori e Adulti hanno una funzione di controllo sia sulla vita in generale, che quindi sugli acquisti on-line. Perciò andremo a segmentare e ispezionare questa categoria al fine di ottenere informazioni utili allo sviluppo di un buon sito.
   Sono stati individuati due criteri di segmentazione:
   \begin{enumerate}[label=\alph*.]
   \item \textbf{Sesso femminile}\\
   Generalmente il membro del nucleo familiare che pone più attenzione allo stile o a come si veste la propria figlia è sicuramente la madre.\\ Perciò potrebbe essere importante sviluppare un sito che le rassicuri e che venda un abbigliamento consono all'età delle proprie figlie, cioè che non offra prodotti troppo provocanti.
   \item \textbf{Reddito medio}\\
   Vista l'età delle ragazze, non sono ancora indipendenti dal punto di vista economico, quindi sono vincolate dal reddito dei propri genitori per l'acquisto di abbigliamento e accessori.\\
   Considerando che generalmente le famiglie con reddito elevato si rivolgono a negozi fisici e boutique o a siti di alta moda con marchi costosi, mentre le famiglie con reddito molto basso è improbabile che facciano acquisti online, è più facile che chi ha un reddito medio sia più attratto da siti di abbigliamento dal costo non troppo elevato.\\
   Il nostro sito si rivolgerà quindi a famiglie con reddito medio e offrirà prodotti alla portata del loro potere d'acquisto.
   \end{enumerate}
 \end{enumerate}
 Un aspetto importante da non sottovalutare è la conoscenza della lingua italiana da ragazze e relativi genitori. Sarà quindi da valutare la possibilità di offrire un modo per tradurre il sito in un'atra lingua.
\section{Ricerca sugli utenti}
A segmentazione completata, per poter ottenere maggiori informazioni riguardo gli utenti del sito, è stato somministrato un sondaggio alle ragazze di età compresa tra i 13 e i 16 anni. Lo scopo di tale sondaggio è stato quello di poter sondare gli interessi nonchè gusti delle ragazze per poter capire meglio i loro bisogni.\\ Il sondaggio è stato creato attraverso un Google form disponibile al seguente indirizzo \href{run:https://goo.gl/forms/qz0KkACAh4ynxCwv2}{https://goo.gl/forms/qz0KkACAh4ynxCwv2}.
Va precisato che il sondaggio è in forma anonima ed è stato somministrato solamente alla segmentazione di ragazze in quanto sono gli utenti principali ed è importante ai fini di una buona progettazione capire su quali prodotti e argomenti si focalizza la loro attenzione.\\
Il sondaggio è stato completato da 40 ragazze; tale numero di persone non può essere considerato un campione valido a livello statistico ma fornisce comunque indicazioni per una miglior progettazione. Le domande sono state suddivise in due sezioni: la prima parte riguarda le informazioni personali, mentre la seconda riguarda gli aspetti relativi all'ambiente della moda.
\subsection{Sondaggio}
\subsubsection{Sezione personale}
Nelle prime domande ci siamo concentrati su dati personali quali l'età, lo stato di provenienza, il luogo di abitazione e la possibilità che le ragazze siano figlie uniche o con sorelle. Di seguito un po di risultati relativi.
\begin{flushleft}
  \includegraphics[width=0.67\textwidth]{"Images Latex/Grafici Sondaggi/1 - Quanti anni hai"}
\end{flushleft}
\begin{flushleft}
  \includegraphics[width=0.75\textwidth]{"Images Latex/Grafici Sondaggi/2 - Dove abiti"}
\end{flushleft}
\begin{flushleft}
  \includegraphics[width=0.75\textwidth]{"Images Latex/Grafici Sondaggi/3 - Sei figlia unica"}
\end{flushleft}
\begin{flushleft}
  \includegraphics[width=0.9\textwidth]{"Images Latex/Grafici Sondaggi/4 - Dove sei nata"}
\end{flushleft}
Possiamo notare che le ragazze che hanno partecipato al sondaggio sono principalmente 15enni/16enni provenienti in modo equo da zone diverse (come città, periferia o campagna). Molte sono figlie uniche e quindi potrebbero essere supportate più facilmente dai propri genitori dal punto di vista economico. La maggior parte sono nate in italia, ma la presenza di un terzo di ragazze di origine straniera potrebbe richiedere l'implementazione di un modo per tradurre il sito almeno in inglese.\\
Sempre stando sulle domande personali abbiamo sondato la loro presenza sui social e internet in generale, concentrandoci prima sui social preferiti e il tempo passato per farci un'idea di quali siano le loro principali fonti di ispirazione sul web e poi vagliando la loro esperienza passata di acquisti online.
\begin{flushleft}
  \includegraphics[width=0.9\textwidth]{"Images Latex/Grafici Sondaggi/5 - Social piu attivi"}
\end{flushleft}
\begin{flushleft}
  \includegraphics[width=0.9\textwidth]{"Images Latex/Grafici Sondaggi/6 - Tempo medio Social"}
\end{flushleft}
Va evidenziato che i social preferiti dalle ragazze intervistate comprendono quasi all'unanimità Instagram (34 persone su 40) e che la maggior parte passa gran parte della giornata su di essi (65\% delle ragazze oltre le 3-4 ore di media).
\begin{flushleft}
  \includegraphics[width=0.9\textwidth]{"Images Latex/Grafici Sondaggi/7 - Esperienza acquisti online"}
\end{flushleft}
Con quest'ultima domanda abbiamo concluso che un portale di e-commerce per ragazze nel nostro target dovrebbe concentrarsi più sulla vendita di vestiti e accessori, preferiti a calzature e trucchi e che c'è un 12\% che non ha mai avuto esperienze di acquisti online e quindi andranno guidate e seguite particolarmente.
\newpage
\subsubsection{Sezione sulla moda}
In questa sezione le domande si sono concentrate sul capire lo stile e i prodotti desiderati e anche da dove le ragazze prendono ispirazione per devidere (Influencer e riviste di moda).
\begin{flushleft}
  \includegraphics[width=0.75\textwidth]{"Images Latex/Grafici Sondaggi/8 - Hai un tuo stile"}
\end{flushleft}
\begin{flushleft}
  \includegraphics[width=1\textwidth]{"Images Latex/Grafici Sondaggi/9 - Interessi prodotti"}
\end{flushleft}
Le ragazze intervistate fanno parte di quella fascia di età ancora un po confusa sul proprio stile e su come vogliono apparire, quindi possiamo notare una divisione equilibrata tra chi ha le idee chiare in fatto di moda e chi ancora non ha ancora elaborato un suo stile. Online vorrebbero tutte comprare principalmente vestiti e quindi è necessario fornire i giusti strumenti per navigare tra tutti i capi in vendita, quali filtri in base ai più disparati criteri (colori, taglie disponibili, estate/inverno) e  magari offrire raccolte di abbinamenti per stile per chi non è ancora sicura di come vestirsi.
\begin{flushleft}
  \includegraphics[width=0.8\textwidth]{"Images Latex/Grafici Sondaggi/10 - Segui Influencer"}
\end{flushleft}
\begin{flushleft}
  \includegraphics[width=0.8\textwidth]{"Images Latex/Grafici Sondaggi/11 - Segui riviste"}
\end{flushleft}
Da questi ultimi due domande, si evince che ormai i punti di riferimento in fatto di moda si sono spostati dalle riviste cartacee specializzate verso il mondo dei social e degli influencer o fashion blogger. La maggior parte delle intervistate hanno confermato di seguire una o più celebrità su internet (come Chiara Ferragni, Chiara Biasi, Chiara Nasti, Beatrice Valli e altre famose influencer); Le riviste ormai sono seguite da poche ragazze e hanno un trend quasi opposto rispetto alle influencer.\\
\subsubsection{Stili}
Dai sondaggi effettuati, abbiamo estratto i principali stili diffusi tra le ragazze intervistate, i quali verranno usati per integrare la sezione di filtro del sito:
\begin{itemize}
  \item Casual
  \item Sportivo
  \item Elegante
  \item Classico
\end{itemize}
Alcuni risultati degli stili (dark, trasandato, androgino) risultano meno comuni e pertanto non verranno considerati nella progettazione di filtri e stili per il sito.
\chapter{Studio di fattibilità}
In questo capitolo andremo ad analizzare la realtà da vicino al fine di poter stabilire se e come sviluppare il progetto. Il primo passo sarà quello di inquadrare il contesto d'uso e i vincoli che ne derivano, per poi identificare determinati profili (\textit{"personas"}) utili a riprodurre il prototipo di utenti ai quali ci rivolgiamo. Successivamente andranno popolati gli scenari, ovvero esempi di come ci si aspetta che gli utenti portano a termine i loro task in un contesto specifico.
\section{Contesto d'uso}
\subsection{Vincoli ambientali}
Al giorno d'oggi con la diffusione degli smartphone l'acquisto di prodotti online è diventata un'attività quotidiana che può essere eseguita ovunque ci troviamo. Però nel caso di acquisti di capi d'abbigliamento si presuppone che ciò avvenga in un ambiente tranquillo dove ci si può concentrare, quale la propria casa o ambienti simili. Quest' ambiente non impone vincoli particolari.\\
Diversamente, il completamento dell'acquisto da parte del genitore può avvenire in qualsiasi momento, purchè essi abbiamo già associato il loro account a quello della propria figlia. L'adulto quindi avrà come vincolo il dover associare il proprio account a quello della figlia, in modo da visualizzare il carrello e accettare o meno di acquistare i prodotti scelti.
\subsection{Vincoli tecnici}
Il primo essenziale vincolo tecnico è possedere una connessione Internet. L'interfaccia verrà implementata utilizzando librerie e tecnologie supportate sia da dispositivi fissi (PC) nonchè mobile, quali smartphone e tablet (quest'ultimi utilizzati maggiormente dal target degli utenti). Per ottimizzare i costi si utilizzerà uno sviluppo del sito con tecniche responsive in modo da adattarsi ad una molteplicità di dispositivi. Inoltre la protezione dei dati personali nonchè sensibili necessita di sistemi di sicurezza così come la gestione dei pagamenti e le sue modalità (PayPal, carta di credito).
\section{Personas}
Sono stati pensati alcuni profili che identificano diversi target possibili per il nostro sistema dagli utenti che abbiamo tracciato.\\
Per ogni \textit{personas} è stato descritto il carattere, gli hobby e le aspirazioni dal punto di vista di stile e moda; inoltre è stato prodotto un grafico che rappresenta competenze e abilità in vari campi:
\begin{itemize}
  \item Competenze tecniche
  \item Competenze di dominio
  \item Competenze linguistiche
  \item Capacità fisiche
  \item Cotivazione
  \item Concentrazione
\end{itemize}
In questo tipo di grafico, più la zona colorata è estesa, più vi è una mancanza o lacuna verso quella competenza. Visto da questo punto di vista è abbastanza intuitivo per individuare in quali competenze e abilità il soggetto eccelle (soglia dell'area colorata più verso il centro) e in quale scarseggia (soglia dell'area colorata più verso l'esterno).
\subsection{Personaggi primari}
\subsection*{Giorgia}
\begin{wrapfigure}{r}{0.3\textwidth}
  \centering
    \includegraphics[width=0.3\textwidth]{"Images Latex/Personas/Giorgia"}
\end{wrapfigure}
\textbf{Demografia}\\
\indent \textbf{Età}: 15 anni\\
\indent \textbf{Residenza}: Milano, Lombardia\\
\indent \textbf{Famiglia}: Figlia unica\\
\indent \textbf{Occupazione}: Studentessa\\
\textbf{Obiettivo}: Rimanere aggiornata sul mondo della moda\\
\textbf{Abitudini}: Uscire con le amiche\\
\textbf{Profilo tecnico}: Frequentare molti social network\\
\rule{\textwidth}{0.5pt}
\subsection*{Arianna}
\begin{wrapfigure}{r}{0.3\textwidth}
  \centering
  \includegraphics[width=0.3\textwidth]{"Images Latex/Personas/Arianna"}
\end{wrapfigure}
\textbf{Demografia}\\
\indent\textbf{Età}: 14 anni\\
\indent\textbf{Residenza}: Ancona, Marche\\
\indent\textbf{Famiglia}: Ha una sorella più piccola\\
\indent\textbf{Occupazione}: Studentessa\\
\textbf{Obiettivo}: Conoscere il mondo della moda ed acquisire un proprio stile\\
\textbf{Abitudini}: Giocare a pallavolo\\
\textbf{Profilo tecnico}: Presente nei principali social network. Utilizzo consapevole di internet\\
\rule{\textwidth}{0.5pt}
\subsection*{Jare}
\begin{wrapfigure}{r}{0.3\textwidth}
  \centering
    \includegraphics[width=0.3\textwidth]{"Images Latex/Personas/Jare"}
\end{wrapfigure}
\textbf{Demografia}\\
\indent \textbf{Età}: 13 anni\\
\indent \textbf{Residenza}: Bologna, Emilia-Romagna\\
\indent \textbf{Famiglia}: Due sorelle maggiori\\
\indent \textbf{Occupazione}: Studentessa\\
\textbf{Obiettivo}: Comprare abbigliamento comodo e utile\\
\textbf{Abitudini}: Praticare atletica a livello agonistico\\
\textbf{Profilo tecnico}: Non molto portata nell'utilizzo della tecnologia\\
\rule{\textwidth}{0.5pt}
\subsection*{Sara}
\begin{wrapfigure}{r}{0.3\textwidth}
  \centering
  \includegraphics[width=0.3\textwidth]{"Images Latex/Personas/Sara"}
\end{wrapfigure}
\textbf{Demografia}\\
\indent\textbf{Età}: 15 anni\\
\indent\textbf{Residenza}: Torino, Piemonte\\
\indent\textbf{Famiglia}: Un fratello più grande\\
\indent\textbf{Occupazione}: Studentessa\\
\textbf{Obiettivo}: Tenere costantemente aggiornato il suo guardaroba\\
\textbf{Abitudini}: Essere attiva sui social e fare photoshoot\\
\textbf{Profilo tecnico}: Utilizzo perfetto di social e relativi dispositivi\\

\subsection{Personaggi Secondari}
\subsection*{Emanuele}
\begin{wrapfigure}{r}{0.3\textwidth}
  \centering
    \includegraphics[width=0.3\textwidth]{"Images Latex/Personas/Emanuele"}
\end{wrapfigure}
\textbf{Demografia}\\
\indent \textbf{Età}: 48 anni\\
\indent \textbf{Residenza}: Torino, Piemonte\\
\indent \textbf{Famiglia}: Moglie e due figli\\
\indent \textbf{Occupazione}: Amministratore delegato\\
\textbf{Obiettivo}: Accontentare sua figlia\\
\textbf{Abitudini}: Molto impegnato e sempre in viaggio per lavoro\\
\textbf{Profilo tecnico}: Sa usare bene PC e Internet. Problemi di vista con piccoli schermi\\
\rule{\textwidth}{0.5pt}
\subsection*{Antonella}
\begin{wrapfigure}{r}{0.3\textwidth}
  \centering
    \includegraphics[width=0.3\textwidth]{"Images Latex/Personas/Antonella"}
\end{wrapfigure}
\textbf{Demografia}\\
\indent \textbf{Età}: 45 anni\\
\indent \textbf{Residenza}: Milano, Lombardia\\
\indent \textbf{Famiglia}: Marito e due figli\\
\indent \textbf{Occupazione}: Professoressa di inglese\\
\textbf{Obiettivo}: Acquistare vestiti consoni all'età della figlia\\
\textbf{Abitudini}: Lavorare a scuola la mattina, e tenersi aggiornata in fatto di moda\\
\textbf{Profilo tecnico}: Utilizzo basilare della tecnologia\\

\section{Goal degli utenti}
\subsection*{Goal delle ragazze}
\begin{enumerate}
  \item Acquisire padronanza nell'utilizzo del sistema
  \item Trovare l'abito che più le piace
  \item Acquisire consapevolezza nel proprio stile
  \item Essere alla moda
  \item Sentirsi uniche e valorizzate
  \item Poter scegliere tra più alternative di stili
  \item Ricevere il prodotto nel minor tempo possibile
  \item Riuscire a trovare ciò che le piace nel minor tempo possibile
  \item Condividere le esperienze dell'uso della piattaforma con soddisfazione
\end{enumerate}
\subsection*{Goal dei genitori}
\begin{enumerate}
  \item Acquistare capi d'abbigliamento per le proprie figlie
  \item Sentirsi al controllo della fase di acquisto e di navigazione
  \item Acquistare prodotti di qualità al minor prezzo possibile
  \item Trovare facilmente le informazioni circa modalità di pagamento e spedizione
  \item Effettuare l'operazione nel minor tempo possibile
  \item Sentirsi sicuro durante la fase di pagamento
  \item Far felici le proprie figlie
  \item Conoscere lo stato dell'ordine
  \item Avere informazioni circa la provenienza dei capi d'abbigliamento
  \item Sicurezza che non siano presenti sul sito capi sessualmente espliciti
  \item Essere rapidi per non sottrarre troppo tempo ad altre attività
\end{enumerate}

\section{Task}
I task rappresentano le operazioni che gli utenti compiono durante il normale utilizzo del sito. I task possibili all'interno di Little Women sono stati identificati dopo aver analizzato i bisogni e obiettivi degli utenti nonchè le risposte emerse dai sondaggi. Alcuni task possono essere realizzati da entrambe le categorie di utenti mentre altri solo da parte degli adulti/genitori. \\
Successivamente tali task verranno utilizzati per somministrare i test agli utenti per valutare i punti di forza e i punti critici del sito web.\\
Di seguito tutti i task possibili in LittleWomen:
\begin{itemize}
  \item \textbf{Task 1} - Registrazione
  \item \textbf{Task 2} - Login
  \item \textbf{Task 3} - Logout
  \item \textbf{Task 4} - Recupero credenziali di accesso
  \item \textbf{Task 5} - Gestione informazioni dell'account personale
  \item \textbf{Task 6} - Ricerca di un prodotto
  \item \textbf{Task 7} - Visualizzazione di un prodotto in base a filtri/categorie (stile, tipo prodotto, etc.)
  \item \textbf{Task 8} - Visualizzare i dettagli di un articolo
  \item \textbf{Task 9} - Selezionare taglia e colore di un articolo
  \item \textbf{Task 10} - Aggiungere un prodotto al carrello
  \item \textbf{Task 11} - Rimuovere un prodotto dal carrello
  \item \textbf{Task 12} - Aggiungere/rimuovere un prodotto alla lista dei desideri
  \item \textbf{Task 13} - Rimuovere un prodotto dalla lista dei desideri
  \item \textbf{Task 14} - Inserire/modificare le informazioni di pagamento
  \item \textbf{Task 15} - Inserire/modificare le informazioni di spedizione
  \item \textbf{Task 16} - Visualizzare gli articoli nel carrello
  \item \textbf{Task 17} - Acquistare il prodotto completando il pagamento
  \item \textbf{Task 18} - Collegare l'account adulto all'account "young"
  \item \textbf{Task 19} - Condividere il sito con i social network
  \item \textbf{Task 20} - Visualizzare informazioni su costi e tempi di spedizione
  \item \textbf{Task 21} - Contattare l'assistenza
  \item \textbf{Task 22} - Visualizzare lo stato di avanzamento dell'ordine
  \item \textbf{Task 23} - Visualizzare gli ordini effettuati
  \item \textbf{Task 24} - Avviare procedura per effettuare un reso
  \item \textbf{Task 25} - Modificare la lingua del sito
  \item \textbf{Task 26} - Inserire feedback e recensioni
\end{itemize}
\section{Scenari}
\subsection*{Giorgia}
Tra una decina di giorni ci sarà la festa di fine anno scolastico nella palestra della scuola di Giorgia. Giorgia è considerata tra le ragazze più carine dell'istituto, ed essendone consapevole, vuole trovare un vestito nuovo che sia unico nella sua eleganza e che la faccia risaltare tra tutte le ragazze. Essendo un'assidua frequentatrice di social network, quali soprattutto Instagram, ha notato ultimamente nel suo news feed un nuovo portale di e-commerce di vestiti per teenagers di nome LittleWomen. Così, un pomeriggio nella sua cameretta prende il suo iPad e inizia ad esplorare il sito alla ricerca di un vestito. Viene subito attratta dalla categoria "Vestiti eleganti" e ne trova uno nero molto dettagliato di cui se ne innamora. Lo aggiunge al carrello e chiede a sua madre di completare l'acquisto.
\subsubsection*{Task associati a Giorgia}
\begin{itemize}
\item \textbf{Task 1} - Registrazione
\item \textbf{Task 2} - Login
\item \textbf{Task 7} - Visualizzazione di un prodotto in base a filtri/categorie
\item \textbf{Task 8} - Visualizzare i dettagli di un articolo
\item \textbf{Task 9} - Selezionare taglia e colore di un articolo
\item \textbf{Task 10} - Aggiungere un prodotto al carrello
\item \textbf{Task 16} - Visualizzare gli articoli nel carrello
\item \textbf{Task 3} - Logout
\end{itemize}
\subsection*{Antonella}
Antonella, dopo aver lasciato che sua figlia utilizzasse l'iPad per scegliere il vestito per la festa, viene avvisata da Giorgia che ha terminato la ricerca e ha trovato una vestito nero di cui si è innamorata. A questo punto, Antonella si mette al computer affiancata da Giorgia. La mamma esamina attentamente il vestito, ma lo trova un po' troppo corto. Nonostante la figlia non sia d'accordo riesce comunque a convincerla a rinunciare all’acquisto di quel vestito, sostituendolo con un altro dallo stesso stile ma un po' più lungo. Infine Antonella conferma l’acquisto del vestito.
\subsubsection*{Task associati a Antonella}
\begin{itemize}
\item \textbf{Task 1} - Registrazione
\item \textbf{Task 2} - Login
\item \textbf{Task 18} - Collegare l'account adulto all'account "young"
\item \textbf{Task 16} - Visualizzare gli articoli nel carrello
\item \textbf{Task 8} - Visualizzare i dettagli di un articolo
\item \textbf{Task 11} - Rimuovere un prodotto dal carrello
\item \textbf{Task 7} - Visualizzazione di un prodotto in base a filtri/categorie
\item \textbf{Task 9} - Selezionare taglia e colore di un articolo
\item \textbf{Task 10} - Aggiungere un prodotto al carrello
\item \textbf{Task 14} - Inserire/modificare le informazioni di pagamento
\item \textbf{Task 15} - Inserire/modificare le informazioni di spedizione
\item \textbf{Task 20} - Visualizzare informazioni su costi e tempi di spedizione
\item \textbf{Task 17} - Acquistare il prodotto completando il pagamento
\end{itemize}
\subsection*{Sara}
Dato che la principale passione di Sara è curare la propria immagine sui social, una delle attività che compie più spesso oltre allo studio nell'istituto di moda è farsi fare delle fotografie dal fratello Massimiliano appassionato di fotografia. Così hanno deciso per il fine settimana di andare nella casa in montagna, non troppo lontano da Torino, per fare qualche foto da caricare su Instagram.\\
Sara è molto esperta in questo settore e decide per questa occasione di comporre qualche outfit sul sito LittleWomen dal suo nuovo iPhone X. Passa un bel pomeriggio a spulciare le categorie e i vari stili analizzando tessuti e trame nel dettaglio per i vari prodotti. A fine pomeriggio il suo carrello contiene un totale di 6 articoli, per comporre 3 outfit differenti. A questo punto chiama il padre per completare il processo di acquisto. Essendo al lavoro, il padre non risponde e quindi gli lascia un messaggio su whatsapp.
\subsubsection*{Task associati a Sara}
\begin{itemize}
\item \textbf{Task 2} - Login
\item \textbf{Task 7} - Visualizzazione di un prodotto in base a filtri/categorie (molte volte)
\item \textbf{Task 8} - Visualizzare i dettagli di un articolo
\item \textbf{Task 9} - Selezionare taglia e colore di un articolo
\item \textbf{Task 10} - Aggiungere un prodotto al carrello
\item \textbf{Task 11} - Rimuovere un prodotto dal carrello
\item \textbf{Task 16} - Visualizzare gli articoli nel carrello
\item \textbf{Task 3} - Logout
\end{itemize}

\subsection*{Emanuele}
Emanuele è in Germania per lavoro e mentre è in riunione riceve una chiamata dalla propria figlia, ma non risponde. A riunione terminata si ritrova un messaggio su whatsapp da sua figlia Sara, che le chiede per l'ennesima volta di completare acquisti online. Essendo in giro per Berlino, cerca di non perder tempo davanti ad uno smartphone in quanto non è troppo pratico coi piccoli schermi. Dato che però alla fine non riesce mai a dir di no alla figlia, arrivato in hotel apre il suo pc e va sul sito indicato dalla figlia. Visto l'utilizzo quotidiano del PC e il fatto che non è la prima volta che entra su LittleWomen, riesce facilmente a fare il login con l'account da genitore e completare l'acquisto dopo aver notato che per l'ennesima volta sua figlia ha esagerato con le quantità e i prezzi. Chiude un occhio e la accontenta comunque.
\subsubsection*{Task associati a Emanuele}
\begin{itemize}
\item \textbf{Task 2} - Login
\item \textbf{Task 16} - Visualizzare gli articoli nel carrello
\item \textbf{Task 14} - Inserire/modificare le informazioni di pagamento
\item \textbf{Task 15} - Inserire/modificare le informazioni di spedizione
\item \textbf{Task 20} - Visualizzare informazioni su costi e tempi di spedizione
\item \textbf{Task 17} - Acquistare il prodotto completando il pagamento
\end{itemize}

\subsection*{Jare}
Durante le lezioni a scuola, dopo aver finito le ore di ginnastica, mentre la prof di italiano sta spiegando, Jare si distrae e chiacchierando con una sua vicina di banco viene a conoscenza del sito LittleWomen molto utilizzato dall'amica. Tornata a casa nel pomeriggio, nota che i pantaloncini con cui fa ginnastica si sono strappati e quindi decide di acquistarne un paio nuovo. Sceglie di utilizzare il sito di cui aveva parlato la sua amica e chiede alla madre di farle utilizzare il suo computer. Jare quindi, affiancata dalla madre, inizia a cercare su LittleWomen un pantaloncino per i suoi allenamenti di atletica. Vorrebbe comprare l'ultimo modello di shorts appena usciti tra le novità, ma la madre nota nella sezione offerte un paio di pantaloncini più sobri e a un prezzo inferiore. Jare si fa convincere facilmente dalla madre e insieme completano l'acquisto.
\subsubsection*{Task associati a Jare}
\begin{itemize}
\item \textbf{Task 1} - Registrazione
\item \textbf{Task 2} - Login
\item \textbf{Task 6} - Ricerca di un prodotto
\item \textbf{Task 7} - Visualizzazione di un prodotto in base a filtri/categorie
\item \textbf{Task 8} - Visualizzare i dettagli di un articolo
\item \textbf{Task 9} - Selezionare taglia e colore di un articolo
\item \textbf{Task 10} - Aggiungere un prodotto al carrello
\item \textbf{Task 16} - Visualizzare gli articoli nel carrello
\item \textbf{Task 20} - Visualizzare informazioni su costi e tempi di spedizione
\item \textbf{Task 22} - Visualizzare lo stato di avanzamento dell'ordine
\end{itemize}

\subsection*{Arianna}
La settimana scorsa Arianna nel suo pomeriggio libero dagli allenamenti decide di fare qualche acquisto su LittleWomen per rinnovare il suo stile. Non essendo ancora arrivati gli ordini, accede col suo account al sito e va a visualizzare la sezione degli ordini effettuati. In seguito visualizza lo stato di avanzamento di ogni ordine, e scopre che uno dei due articoli in arrivo ha la taglia errata, ma non sapendo come bloccare l'ordine chiude tutto ed esce con le amiche.
Il giorno dopo arrivano gli ordini e quindi ritorna sul sito per cercare di fare il reso dell'articolo sbagliato. Dopo un po' di tempo riesce a capire come avviare il reso, completa la procedura e prepara il pacco da rispedire.
\subsubsection*{Task associati a Arianna}
\begin{itemize}
\item \textbf{Task 2} - Login
\item \textbf{Task 23} - Visualizzare gli ordini effettuati
\item \textbf{Task 22} - Visualizzare lo stato di avanzamento dell'ordine
\item \textbf{Task 24} - Avviare procedura per effettuare un reso
\item \textbf{Task 3} - Logout
\end{itemize}
\chapter{Valutazione delle risorse esistenti}
\section{Expert usability review}
I siti di e-commerce dedicati o con una sezione dedicata alle ragazzine dai 13 ai 16 anni sono molteplici. Tra questi abbiamo preso in considerazione Abercrombie \& Fitch (\href{https://www.abercrombie.com}{https://www.abercrombie.com}), il quale si occupa di vendita di vestiti per entrambi i sessi sia adulti che bambini, quindi la nostra valutazione si è concentrata solo sulla parte del che comprende un target di utenti di sesso femminile tra i 13-16 anni.
\subsection{Scelta delle linee guida}
Tra le linee guida esistenti si è deciso di adottare le 10 euristiche di Nielsen in quanto dispone di domande specifiche che permettono di eseguire un'analisi approfondita del sistema.\\
Di seguito le 10 euristiche di Nielsen:
\begin{enumerate}
  \item \textbf{Visibilità dello stato del sistema}\\Il sistema deve fornire feedback all'utente in modo da tenerlo informato su ciò che accade.
  \item \textbf{Corrispondenza tra sistema e mondo reale}\\Il sistema deve essere familiare all'utente, parlare il suo stesso linguaggio e seguire convenzioni del mondo reale.
  \item \textbf{Controllo e libertà}\\L'utente deve avere il controllo del contenuto informativo e muoversi liberamente tra gli argomenti. Il sistema deve fornire "uscite di sicurezza".
  \item \textbf{Consistenza e standard}\\L’utente deve aspettarsi che le convenzioni del sistema siano valide per tutta l’interfaccia.
  \item \textbf{Prevenzione dell'errore}\\Evitare di porre l’utente in situazione ambigue, critiche e che possono portare all’errore.
  \item \textbf{Riconoscimento anziché ricordo}\\Le istruzioni per l’uso del sistema devono essere ben visibili e facilmente recuperabili.
  \item \textbf{Flessibilità d’uso}\\Offrire all’utente la possibilità di un uso differenziale (a seconda della sua esperienza) dell’interfaccia.
  \item \textbf{Design e estetica minimalista}\\Dare maggior importanza al contenuto che all’estetica. Ogni informazione aggiuntiva diminuisce la visibilità.
  \item \textbf{Aiuto all’utente}\\Aiutare l’utente a riconoscere, diagnosticare e recuperare l’errore.
  \item \textbf{Documentazione}\\Anche se il sistema dovrebbe essere usabile senza documentazione è preferibile che essa sia disponibile.
\end{enumerate}
\subsection{Prima ispezione del Sistema}
L'ispezione iniziale è avvenuta solo sulle sezioni interessanti per il nostro target, quindi ci siamo concentrati sulla sezione "kids" sottocategoria "girls" e tutto ciò che ruota attorno all'acquisto.
Ci siamo concentrati per esempio su:
\begin{itemize}
  \item visualizzare i vestiti tramite l'utilizzo di filtri
  \item visualizzare dettagli di un articolo
  \item acquisto di un articolo e spedizione
  \item etc...
\end{itemize}
Le varie pagine delle collezioni sono tutte strutturate sotto forma di vetrina virtuale (rettangoli con foto, nome articolo e prezzo) con sfondo grigio. Su ogni pagina di questo tipo possiamo trovare o impostare alcuni filtri per raffinare la ricerca.\\
Ovviamente è presente anche la pagina relativa al carrello e all'acquisto e spedizione di prodotti, impostate tutte in modo chiaro e semplice da usare.\\
In generale dopo la prima analisi possiamo dire che il sito presenta un design pulito ed essenziale che risalta i prodotti in vendita e ne permette una facile visualizzazione e acquisto.
\subsection{Analisi Diretta: Sistema vs Linee Guida}
Durante l'analisi diretta, il sito non mostra evidenti problemi gravi di usabilità secondo le euristiche applicate. Comunque alcune cose potevano venir sviluppate meglio:
\begin{itemize}
  \item Alcuni termini utilizzati relativi ai prodotti o filtri sono specifici del dominio e forse potrebbero non essere chiari per un utente medio.\\
  (Euristica non rispettata: 2)
  \item Nella barra a lato sono presenti due sottomenù entrambi chiamati Collezioni, non si capisce qual'è la differenza tra le due.\\
  (Euristica non rispettata: 8)
  \item Non si riesce a capire facilmente dove è possibile registrare un account per fare acquisti sul sito. Inoltre la registrazione è avvenuta quasi per puro caso sentendosi quasi persi in questa fase.\\
  (Euristiche non rispettate: 1, 5)
  \item l'impostazione del filtro per negozio non funziona bene. Cliccando su "Vicino a Me" non mostra neanche un risultato. Vi è anche una discrepanza tra impostazione tramite Nome di Città e CAP: inserendo un CAP e poi una città, nel momento in cui il CAP si riferisce a un luogo più distante dallo store rispetto alla città inserita, nel caso del CAP da un risultato mentre con la città no.\\
  (Euristiche non rispettate: 4, 9, 5)
  \item Il footer del sito presenta troppe informazioni scritte in poco spazio che spingono l'utente ad evitarlo, nonostante ci siano alcuni link utili quali: assistenza ordini, resi online, stato dell'ordine.\\
  (Euristiche non rispettate: 8, 6)
  \item In fase di aggiunta al carrello, il sito richiede di selezionare taglia e vestibilità. L'errore riportato dalla mancata selezione della vestibilità non è chiaro, in quanto mostra un errore relativo alla mancata selezione della taglia.\\
  (Euristiche non rispettate: 9)
  \item In fase di pagamento, con accesso all'account effettuato, compaiono alcune scritte in inglese anche se la lingua selezionata è l'italiano. Si può intuire a fatica che si tratta di un buono sconto.\\
  (Euristiche non rispettate: 2)
\end{itemize}
\subsection{Analisi inversa: Linee Guida vs Sistema}
Eseguendo l'analisi inversa sono state prese le euristiche di Nielsen sopra citate e sono state confrontate con il sistema. Di seguito sono riportate le euristiche che non sono state rispettate dal sistema. I problemi riscontrati nell'analisi diretta sono stati omessi.
\begin{itemize}
  \item \textbf{Euristica 3} - Il sistema non pone l'utente nella condizione di muoversi liberamente, poichè dopo aver selezionato un articolo non vi è un modo semplice e intuitivo per poter tornare alla sezione precedente. L'unico modo possibile è attraverso un piccolo menu orizzontale e poco visibile.
  \item \textbf{Euristica 10} - Il sito non presenta alcuna forma di documentazione.
\end{itemize}
\section{User testing}
\subsection{Protocollo di testing}
Dato che non è possibile migliorare iterativamente questo sito sotto esame, abbiamo applicato una variante del Discount usability  test (proposto da J. Nielsen nel 1994).\vspace{5mm}
\\\vspace{5mm}
\textbf{Metodo di testing}: Variante del Discount usability test
\\\vspace{5mm}
\textbf{Metodologia di testing}: Thinking Aloud
\\\vspace{5mm}
\textbf{Elenco dei task da testare}: Sono stati selezionati più task da quelli offerti dal sito che messi insieme permettono di completare un'operazione:
  \begin{itemize}
    \item \underline{\textit{Task 1}}: Trovare una felpa chiara senza scritte
    \item \underline{\textit{Task 2}}: Effettuare la registrazione e il Login
    \item \underline{\textit{Task 3}}: Completare tutta la fase d'acquisto
  \end{itemize}
  \textbf{Soggetti}:
  \begin{itemize}
    \item Federica, studentessa di 20 anni, pratica con la tecnologia;
    \item Michela, tatuatrice di 24 anni, mediamente abile nell'uso della tecnologia;
    \item Daniele, impiegato di 56 anni, poco pratico nell'utilizzo della tecnologia.
  \end{itemize}
  \textbf{Descrizione dei risultati attesi}: per effettuare una stima dei risultati otteuti sono stati applicati i criteri EEE (Efficacia, Efficienza, Emozioni); per efficacia s'intede l'accuratezza con cui gli utenti riescono a raggiungere determinati obiettivi, l'efficienza misura il tempo minimo richiesto per portare a termine un task in maniera efficace, mentre l'emozione indica il comfort e l'atteggiamento positivo dell'utente nell'utilizzo del sistema.\\
  Per ognuno di questi criteri si definiscono queste scale:
  \begin{itemize}
  \item Efficacia:
    \begin{itemize}
    \item Task portato a termine in modo autonomo: \textit{Punteggio 3}
    \item Tesk portato a termine con aiuto: \textit{Punteggio 2}
    \item Task abbandonato: \textit{Punteggio 1}
    \end{itemize}
  \item Efficienza:
    \begin{itemize}
    \item Grado di completamento anche parziale del task dal punto di vista del tempo impiegato. 1 equivale a tempi prolissi, 3 tempi brevi o ragionevoli.
    \end{itemize}
  \item Emozione/Soddisfazione
    \begin{itemize}
    \item domanda secca su quanto ci si ritiene soddifatti nell'utilizzo del sito complessivamente, risposta data con un valore su una scala da 1 a 7.
    \end{itemize}
  \end{itemize}
  \includegraphics[width=1\textwidth]{"Project Management Sources/RisultatiAttesi"}
  \subsection{Analisi dei risultati ottenuti}
  \includegraphics[width=1\textwidth]{"Project Management Sources/RisultatiReali"}\vspace{5mm}
  In linea di massima i valori si discostano poco da quelli attesi, indice che l'affermazione fatta in fase di Expert Usability Review è confermata dagli utenti e che il sito è effettivamente fatto bene. Si possono notare alcune criticità nel portare a termine autonomamente alcuni task, anche da parte di utenti consapevoli della tecnologia utilizata. Come ci si aspettava, per un utente non avvezzo al dominio del sito (Daniele), la navigazione risulta più complessa  e meno soddisfacente.\vspace{5mm}
  Nell'utilizzo del sito in generale gli utenti non hanno riscontrato particolari problematiche. Nell'utilizzo del sito per completare i task, invece, sono sorte alcune criticità:
  \begin{itemize}
    \item \underline{Task 1}: "Trovare una felpa chiara senza scritte"
    \begin{itemize}
    \item La ricerca di un prodotto specifico non è così semplice se non si riesce ad utilizzare i corretti criteri di ricerca, il sito spinge di più ad una visualizzazione passiva di collezioni e sezioni prefiltrate. \\Per utenti non pratici il risultato è stato quello di perdersi all'interno delle collezioni alla ricerca di una felpa senza scritte (\textcolor{red}{E1}).
    \item Utilizzare la barra di ricerca con parole chiave in alto a destra non è stato immediato data la piccola iconcina. \\Sarebbe più facilmente individuabile con una scritta a fianco dell'icona (\textcolor{red}{E2}).
    \end{itemize}
    \item \underline{Task 2}: "Effettuare la registrazione e il Login"
    \begin{itemize}
      \item Senza aiuto esterno sono stati riscontrati grossi problemi ad identificare come effettuare la registrazione al sito. Al posto di una sezione Registrati è presente un voce "iscriviti all'A\&F Club", che manda in confusione e senza aiuto non aiuta ad effettuare la prima registrazione. Sarebbe preferibile sostituirlo con qualcosa di più universalmente riconosciuto come "Registrati". (\textcolor{red}{E3})
    \end{itemize}
    \item \underline{Task 3}: Completare tutta la fase d’acquisto\\
    la fase d'acquisto è fatta tutta abbastanza bene, escluse alcuni dettagli collegati ad essa:
    \begin{itemize}
      \item In fase d'acquisto è possibile selezionare il ritiro in un negozio "Vicino a te". La scelta di questo punto di ritiro ha presentato parecchi problemi e non da sempre almeno un risultato, cosa che manda in confusione. In relatà il risultato viene trovato solo se si rientra in un certo raggio di distanza da un negozio fisico. Sarebbe opportuno mostrare comunque il negozio fisico più vicino con indicati magari i km di distanza. (\textcolor{red}{E4})
    \end{itemize}
  \end{itemize}
  Gli errori riscontrati sono classificati in ordine di gravità nella seguente tabella, secondo la classificazione proposta sempre da Nielsen.\\\\
  \includegraphics[width=1\textwidth]{"Project Management Sources/ImpattoErrori"}
  \newpage
  \subsection{Curva d'urgenza}
  Di seguito è possibile visualizzare il grafico relativo alla curva d'urgenza. Gli errori presenti (differenziati dai vari simboli) sono stati riportati in un grafico bidimensionale "impatto per frequenza" e fanno riferimento alla tabella della pagina precedente.\\\\
  \includegraphics[width=1\textwidth]{"Project Management Sources/CurvaUrgenza"}
\chapter{Proposta di design}

Dato che il nostro obiettivo è soddisfare i bisogni (goal) degli utenti, abbiamo impostato il design seguendo un approccio Goal-oriented. Questo approccio permette di ottimizzare lo sviluppo di un sistema usabile, elimando task irrilevanti nel raggiungimento di obiettivi per gli utenti.\\
Si è deciso di utilizzare il modello CAO=S, che da la possibilità di sviluppare un sistema usabile, minimzzando i costi e gli errori più comuni.\\
\section{Architettura dell'Informazione}
Per strutturare l'informazioni all'interno del nostro sistema abbiamo adottato un approccio dall'alto verso il basso (\textit{Top-Down}).\\
Questo approccio utilizza specifiche domande per analizzare alcuni concetti e aspetti importanti per il nostro sito con l'obiettivo di individuare il modo migliore per comunicare con l'utente permettendo al progettista di adottare una strategia al fine di creare un'interfaccia dove sia semplice e immediato trovare ciò che si cerca o il modo per arrivarci.\\\\\\
Elenchiamo di seguito le domande con considerazione legate al portale che si vuole progettare:
\begin{enumerate}
  \item Dove sono?\\
  In ogni situazione l'utente deve sapere dove si trova, cioè deve sapere che sta navigando all'interno del nostro sito. Tale domanda è soddisfatta posizionando in alto una barra di navigazione con "Logo" e "Nome del sito", fissando tale barra in alto anche nal caso in cui si effettui uno scrolling della pagina. Sempre nella navbar sarà ben evidente in che sezione ci si trova o nel caso i filtri della ricerca.
  \item So cosa sto cercando? Nel caso so dove trovarlo?\\
  Ogni utente deve poter trovare ciò che cerca facilmente in ogni pagina del sito. Questa necessità è soddisfatta da filtri preimpostati in modo chiaro e un campo di ricerca sempre visibile nella barra di navigazione per effettuare ricerche con parole chiave.
  \item Come posso esplorare il sito?\\
  Questo sito ha a disposizione innumerevoli capi d'abbigliamento e bisogna fornire un modo semplice con cui esplorare il sito. Virtualmente la pagina mostrata sarà sempre la stessa, ma cambieranno i prodotti. Per questo motivo vi sarà una barra laterale sempre visibile con dei filtri preimpostati per visualizzare sottogruppi di prodotti mostrati e anche un modo per tornare indietro. Si potranno impostare anche filtri propri.\\
  Durante la visualizzazione di un prodotto comparirà un overlay sulla pagina attuale con dettagli del prodotto, mantenendo l'interfaccia e i prodotti mostrati sotto in dissolvenza. Si potrà tornare facilmente indietro.
  \item Cosa rende questo sito unico?\\
  Ogni sito deve possedere un elemento o un’organizzazione degli elementi che lo contraddistingue dagli altri siti.\\
  Essendo un sito di ecommerce, la sua unicità principale sono i prodotti in vendita e come vengono messe in risalto le ultime novità. La gestione della visualizzazione delle novità da unicià al sito.
  \item Cos'è questo sito e cosa posso trovarci?\\
  Un utente che si trova su questo sito deve poter capire di cosa tratta e che cosa può trovarci. Perciò il sito in esame mostrerà un home page con le ultime novità uscite e collegamenti alle principali collezioni in tendenza in questo periodo tra gli utenti, in modo da fornire un rapido accesso a tutto ciò che il sito ha da offire ad un utente conscio delle ultime tendenze, ma anche ad uno meno informato sul mondo della moda.
  \item Cosa sta succedendo in questo periodo?\\
  Su un sito potrebbe essere necessario mostrare le informazioni relative all'argomento trattato esterne ad esso. Essendo però queto un sito di e-commerce, il fulcro è mostrare e vendere prodotti e una sezione con le ultime news sulla moda andrebbe a distogliere l'utente da ciò che veramente vogliamo che faccia sul sito: visualizzare e comprare prodotti. È comunque presente una sezione novità e tendenze per mostrare i prodotti più in voga del momento.
  \item Danno peso alla mia opinione?\\
  Per un sito potrebbe essere necessario fornire un modo per dare un feedback agli utenti, in questo caso un sistema di recensioni. Per il nostro sito in esame le informazioni date relativamente al prodotto saranno abbastanza esaustive da non richiedere la presenza di una sezione recensioni e feedback sul prodotto.
  \item Come posso ricevere assistenza da un essere umano?\\
  Potrebbe essere utile per un utente parlare direttamente con un essere umano per avere informazioni o assistenza in caso di problemi.\\
  Sarà quindi presente in ogni pagina un bottone in basso a sinistra, per non interferire con la navigazione, con cui poter avviare una chat con un assistente umano.\\
  Dato che il sito rappresenta online una catena di abbigliamento e moda, sarà possibile trovare anche i numeri dei negozi in una sezione chiamata \textit{footer}, posta in basso dato che sono informazioni secondarie e poco usate.
  \item C'è un indirizzo fisico dell'organizzazione?\\
  Riprendendo la domanda precedente, dato che il sito rappresenta online una catena distribuita sul territorio, sono presenti delle sedi fisiche. Questa informazione non è particolarmente importante, e infatti viene abbinata nel footer ai numeri di telefono delle sedi, sempre suddivise per regione.
\end{enumerate}

\section{Modello CAO=S}
Il modello CAO=S è un modello di progettazione goal-oriented. È basato sullo studio dei tipi di informazione (\textit{Concetti}) che l'applicazione deve poter manipolare per conto dei tipi di utenti (\textit{Attori}) fornendo determinati comandi (\textit{Operazioni}). Una corretta analisi di ciò permette di generare tre tipi di \textit{Strutture} gestite dal modello:
\begin{itemize}
  \item Viste:  schermate di proprietà dei concetti
  \item Strutture dati: modelli per la memorizzazione dei concetti
  \item Navigazione:  meccanismo di navigazione da una vista all'altra.
\end{itemize}
Di seguito vengono analizzati nel dettaglio ogni componente del modello.
\subsection{Concetti}
I concetti rappresentano il tipo di informazione gestita dall'applicazione e ciò non si identifica nelle strutture dati ma bensì nel modo in cui gli utenti percepiscono e comprendono l'informazione memorizzata nella struttura dati. Proprio per questo è importante che il sistema riesca ad evitare i problemi di standardizzazione, adottanto un lessico che sia comprensibile a tutti. Per questo si è deciso di utilizzare i seguenti concetti all'interno del sito in modo da evitare fraintendimenti.
\begin{itemize}
  \item Novità
  \item Pezzi di sotto
  \item Abbinamenti
  \item T-shirt con stampa
  \item Carrello
  \item Accessori
  \item Login e Registrazione
  \item Trova un negozio
\end{itemize}
Questi appena descritti sono in principali sostantivi utilizzati nel sito che non richiedono una conoscenza approfondita del dominio applicativo. Ad esempio è stato scelto di differenziare "T-shirt con stampa" piuttosto che "Graphic tees" e "Registrazione" piuttosto che "Entra a far parte del nostro club".
\subsection{Attori}
Il secondo componente del modello CAO=S è rappresentato dagli Attori. In generale gli attori sono le categorie di utenti che agiscono sull'interfaccia dell'applicazione, manipolando le strutture dati per completare i loro task. Gli attori quindi sono caratterizzati da ruoli che ne determinano i task e quindi le operazioni. Nel nostro sistema, in conformità con il modello CAO=S verranno analizzati gli attori sia con una breve descrizione della loro persona e sia attraverso il racconto di uno scenario in cui viene esplicitato il suo obbiettivo accompagnato da un diagramma di strategia che analizza le capacità e competenze di ciascun attore su una scala da 1 a 5 (1 valore basso, 5 valore molto alto).\\

\subsubsection{Giorgia}
\rule{\textwidth}{0.5pt}
\begin{wrapfigure}{r}{0.5\textwidth}
  \centering
    \includegraphics[width=0.3\textwidth]{"Images Latex/Personas/Giorgia"}
\end{wrapfigure}
  Giorgia ha 15 anni e frequenta il liceo socio-economico. Ama uscire e divertirsi con le sue amiche, con le quali condivide la passione per la moda. Naturalmente è un'utente esperta dei social network; tra tutti il suo preferito è Instagram che frequenta assiduamente. Su instagram le fashion blogger sono il suo punto di riferimento e la sua preferita è Chiara Ferragni, di cui ama carattere e personalità, nonchè il suo modo di vestirsi. Infatti Giorgia è sempre alla ricerca di uno stile e di capi che la contraddistinguano dalle sue coetanee, e preferisce uno stile casual che sia al contempo elegante e raffinato.\\
  \begin{center}
    \includegraphics[width=0.6\textwidth]{"Images Latex/Personas/Giorgia15"}
  \end{center}
\newpage
\subsection*{Arianna}
\rule{\textwidth}{0.5pt}
\begin{wrapfigure}{r}{0.5\textwidth}
  \centering
  \includegraphics[width=0.3\textwidth]{"Images Latex/Personas/Arianna"}
\end{wrapfigure}
Arianna ha 14 anni e ha appena iniziato il primo anno di una scuola professionale. Si è appena affacciata al mondo della moda e non è ancora troppo interessata a crearsi uno stile tutto suo, ma comunque attraverso amiche e alcune fashion blogger sui social (Instagram e Snapchat) è aggiornata sulle ultime tendenze. Essendo una millennials non ha problemi nell'utilizzo della tecnologia anche se non è troppo portata e preferisce passare il tempo libero giocando nella squadra di pallavolo del suo paese con ottimi risultati e uscendo con le amiche. Deve ancora inquadrare un suo stile e per lo più si veste casual e sportiva, ma crescendo sta maturando e vorrebbe iniziare a vestirsi un po più elegante.
\begin{center}
  \includegraphics[width=0.7\textwidth]{"Images Latex/Personas/Arianna14"}
\end{center}
\newpage
\subsection*{Jare}
\rule{\textwidth}{0.5pt}
\begin{wrapfigure}{r}{0.5\textwidth}
  \centering
    \includegraphics[width=0.3\textwidth]{"Images Latex/Personas/Jare"}
\end{wrapfigure}
Jare ha 13 anni ed ha origini nigeriane. La sua famiglia è arrivata in Italia quando Jare aveva solo 5 anni, ma nonostante ciò ha una buona padronanza della lingua italiana. Ha un carattere socievole ed estroverso, tanto che a scuola viene spesso sorpresa a chiacchierare con le sue amiche. Non adora molto seguire le regole, con notevoli sforzi anche da parte dei genitori nel renderla disciplinata. La sua passione è l'atletica, che pratica a livello agonistico da ormai qualche anno con ottimi risultati. La tecnologia non è proprio il suo forte, ed è alle prime armi con l'utilizzo di social network e siti di e-commerce. Essendo quindi nuova al mondo della moda, non sa ancora cosa acquistare perciò si lascia condizionare dai propri genitori. L'unica sua richiesta è che l'abbigliamento sia sportivo e soprattutto comodo.
\begin{center}
  \includegraphics[width=0.6\textwidth]{"Images Latex/Personas/Jare13"}
\end{center}
\newpage
\subsection*{Sara}
\rule{\textwidth}{0.5pt}
\begin{wrapfigure}{r}{0.5\textwidth}
  \centering
  \includegraphics[width=0.3\textwidth]{"Images Latex/Personas/Sara"}
\end{wrapfigure}
Sara è una ragazza di 16 anni che frequenta l'istituto di moda. È  da sempre appassionata di questo mondo e infatti fin dai 13 anni seguiva assiduamente le principali influencer come Chiara Ferragni o Veronica Ferraro e ispirandosi a loro ha iniziato a curare il suo volto social soprattutto su Instagram, ma anche sugli altri principali social da giovani (Ask.fm e Snapchat). Ha un discreto successo e sulla sua pagina Instagram vanta quasi 200 mila follower rendendola di fatto una piccola influencer. Ha uno stile ben preciso e definito negli anni, identificabile con un elegante casual molto maturo per la sua età. È un'assidua frequentatrice di siti di e-commerce sia di marche affermate che emergenti da cui compra sempre i migliori capi al miglior prezzo.
\begin{center}
  \includegraphics[width=0.6\textwidth]{"Images Latex/Personas/Sara16"}
\end{center}
\newpage
\subsection*{Emanuele}
\rule{\textwidth}{0.5pt}
\begin{wrapfigure}{r}{0.5\textwidth}
  \centering
    \includegraphics[width=0.3\textwidth]{"Images Latex/Personas/Emanuele"}
\end{wrapfigure}
Abita a Torino, ed è il padre di Sara che ha 16 anni oltre che di Matteo, che ha 23 anni. È amministratore delegato di un'importante azienda farmaceutica; il lavoro lo impegna molto infatti durante la settimana viaggia spesso all'estero ma nel weekend riesce a dedicare un po' di tempo ai propri figli. L’utilizzo principale che fa del PC e di Internet è per mantenersi in contatto con gli impiegati dell'azienda, ulteriori aziende, banche e per controllare il bilancio dell’azienda. A volte ha alcune discussioni con sua figlia in merito di vestiti, infatti viene regolarmente sollecitato per l’acquisto di vestiti. Però riesce quasi sempre a farsi perdonare facendole regali dalle varie parti del mondo in cui viaggia. In fatto di moda non è molto aggiornato, e si concede alle richieste della figlia consentendole di acquistare capi su negozi e-commerce come LittleWomen, di cui è molto fiducioso.
\begin{center}
  \includegraphics[width=0.6\textwidth]{"Images Latex/Personas/Emanuele41"}
\end{center}
\newpage
\subsection*{Antonella}
\rule{\textwidth}{0.5pt}
\begin{wrapfigure}{r}{0.5\textwidth}
  \centering
    \includegraphics[width=0.3\textwidth]{"Images Latex/Personas/Antonella"}
\end{wrapfigure}
Antonella è la mamma di Giorgia, e di Massimiliano, che ha 20 anni. È una professoressa di inglese nelle scuole superiori, ma nel passato ha lavorato come intermediaria linguistica per un'importante azienda di moda estera. Proprio per questo conosce il mondo della moda e spesso affianca sua figlia Giorgia nella ricerca di capi d'abbigliamento dai gusti raffinati che rispettano comunque l'età di sua figlia. Antonella non è esperta di tecnologia; la sua conoscenza si limita ad effettuare ricerche su Google e a controllare la figlia nell'effettuare acquisti online.
\begin{center}
  \includegraphics[width=0.6\textwidth]{"Images Latex/Personas/Antonella45"}
\end{center}
\chapter{Valutazione del design proposto}
\chapter{Conclusioni}
\chapter{Allegati}
\chapter{Licenza}
\end{document}
